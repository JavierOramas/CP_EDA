\documentclass{article}
\title{CP4 EDA árbol binario de búsqueda}
\author{Javier Alejandro Oramas López C212}
\date{}

\begin{document}
    \maketitle
    \section{}
    Ejercicio resuelto previamente
    \section{}
    Sea y un nodo del árbol T y x un nodo hoja hijo de y\\
    Sabemos que si x es hijo de y, x = y.Previous (Right) o x = y.Next (left)\\
    supongamos que x = y.Next (Left)\\
    Como se demostró en el ejercicio 1, si x = y.Next:\\
    y.key > x.key y además se cumple:\\
    \[\not \exists z : y.key > z.key > x.key \]
    por tanto y es el menor elemento de T mayor que x.key
    
    de forma análoga se demuestra que si x = y.Previous (left):\\
    y es el mayor elemento de T menor que x.key
    
    Luego, tenemos que si se cumple que x es un nodo hoja hijo de y, y es el menor elemento de T mayor que x.key o y es el mayor elemento de T menor que x.key
    
    \section{}
    Sea x un nodo de T al que se le aplica la función InOrder\\
    Si $x.Left \not = Null$, todos los elementos de x.Left van a ser menores que x.key\\
    Si $x.Left = Null$, no habrá elementos menores que x.key\\
    de manera análoga, \\
    Si $x.Right \not = Null$, todos los elementos de x.Right van a ser mayores que x.key\\
    Si $x.Right = Null$, no habrá elementos mayores que x.key\\

    luego, el array ordenado ascendentemente de un arbol será:\\

    \[Ordenado(x_0.Left) + x_0 + Ordenado(x_0.Right)\]
    
    que será equivalente a ejecutar el algoritmo InOrder en el árbol

    \section{}
    Ejercicio resuelto previamente

\end{document}