\documentclass[10pt,a4paper]{article}
\usepackage[spanish]{babel}
\usepackage{listings}
\usepackage{color}
\usepackage[utf8]{inputenc}

\definecolor{dkgreen}{rgb}{0,0.6,0}
\definecolor{gray}{rgb}{0.5,0.5,0.5}
\definecolor{mauve}{rgb}{0.58,0,0.82}
\definecolor{pink}{rgb}{1.0,0.0,0.5}
\lstset{
  frame=none,
  language=C++,
  aboveskip=3mm,
  belowskip=3mm,
  showstringspaces=false,
  columns=flexible,
  basicstyle={\small\ttfamily},
  numbers=none,
  numberstyle=\tiny\color{pink},
  keywordstyle=\color{blue},
  commentstyle=\color{dkgreen},
  stringstyle=\color{green},
  breaklines=true,
  breakatwhitespace=true,
  tabsize=3
}

\author{Javier Alejandro Oramas L\'opez C212}
\title{Clase Pr\'actica 1 EDA I}

\begin{document}
    \maketitle
    \section{Ejercicio 1}
        \subsection{ $ 8n^{2} \leq 64n $}
            \[ 8n^{2} \leq 64n \]
            \[ \frac{8n^{2}}{n} \leq 64 \]
            \[ 8n \leq 64\]
            \[ n \leq 8 \]        
         \subsection{$ 100n^2 \leq 2^n $}
            \[ 100n^2 \leq 2^n \]
            \[ 10n \leq 2^{\frac{n}{2}} \]
            \[ n \leq 14 \]            
    \section{Ejercicio 2}
        \subsection{}
            \[ c_{0} + n( c_{1} + c_{2}) +c_{3} \]
            \paragraph{
                T (n) es m\'inimo cuando el primer elemento es mayor que el segundo \\
                T (n) es m\'aximo cuando el array esta totalmente ordenado
                }
        \subsection{ }
            \[ c_{0} + n \left( c_{1} + c_{2} + \sum_{i=2}^{N} i \left(c_{3} +c_{4} + c_{5} +c_{6} +c_{7} \right)\right) \]
            \paragraph{
                T (n) es m\'inimo cuanod el array est\'a totalmente ordenado \\
                T (n) es m\'aximo cuando el array est\'a totalmente desordenado
                }
        \subsection{ }  
            \[ c_{0} + n(c_{1} + c_{2}) + c_{3} \]
            \paragraph{
                T (n) es m\'inimo cuanod el primer elemento es igual a key \\
                T (n) es m\'aximo cuando no contiene a key
                }
    \section{Ejercicio 3}
        \paragraph{
        El tama\~no de la entrada es 1
        }
        \subsection{ }
            \[ c_{0} + c_{1} + c_{2} + \sqrt{x}\left( C_{3} + c_{4}\right) \]
        \subsection{ }
            \[ c_{0} + c_{1} + c_{2} + c_{3} +  x ( c_{4} + c_{5} + c_{6} +c_{7}) + c_{8} \]
            
    \section{Ejercicio 4}
        \begin{lstlisting}
            int Exponentiation(int x, int n)
            {
                if(n == 1) 
                    return x;
                if(n % 2) 
                    return Exponentiation(x,n-1) * x;
                else 
                    return Exponentiation(x,n/2)*Exponentiation(x,n/2)  
            }
        \end{lstlisting}
        
\end{document}
