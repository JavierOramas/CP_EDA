\documentclass[10pt,a4paper]{article}
\usepackage[spanish]{babel}
\author{Javier Alejandro Oramas L\'opez C212}
\title{Clase Pr\'actica 1 EDA I}
\begin{document}
    \maketitle
    \section{Ejercicio 1}
        \subsection{a) $ 8n^{2} \leq 64n $} 
            \begin{enumerate}
                $$ 8n^{2} \leq 64n $$
                $$ \frac{8n^{2}}{n} \leq 64 $$
                $$ 8n \leq 64$$
                $$ n \leq 8 $$
            \end{enumerate}
        
         \subsection{b) $ 100n^2 \leq 2^n $}
            $$ 100n^2 \leq 2^n $$
            $$ 10n \leq 2^{\frac{n}{2}} $$
            $$ n \leq 14 $$
            
    \section{Ejercicio 2}
        T(n) alcanza el m\'inimo valor cuando n = 0
        T(n) alcanza el m\'aximo valor cuando el tama\~no en memoria de A es el m\'aximo posible
        \subsection{a)}
            $$ c_{0} + n( c_{1} + c_{2}) +c_{3} $$
        \subsection{b)}
            $$ c_{0} + n(c_{1} + c_{2} + \sum_i=1^{N} i (c_{3} +c_{4} + c_{5} +c_{6} +c_{7} )) $$
        \subsection{c)}
            $$ c_{0} + n(c_{1} + c_{2}) + c_{3} $$
    \section{Ejercicio 3}
        El tama\~no de la entrada es 1
        \subsection{a)}
            $$ c_{0} + c_{1} + c_{2} + \sqrt{x}( C_{3} + c_{4} $$
        \subsection{b)}
            $$ c_{0} + c_{1} + c_{2} + c_{3} +  x ( c_{4} + c_{5} + c_{6} +c_{7}) + c_{8} $$
            
    \section{Ejercicio 4}
        \code
        int Exponentiation(int x, int n)
        {
            if(n == 1) 
                return x;
            if(n % 2) 
                return Exponentiation( x,n-1) * x;
            else 
                return Exponentiation(x,n/2)*Exponentiation(x,n/2)  
        }
        
\end{document}
