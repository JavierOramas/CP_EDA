\documentclass{article}
\usepackage[utf8]{inputenc}
\author{Javier A. Oramas López}
\title{CP-5 AVL}

\begin{document}
    \maketitle

    \section{ Determine si las siguientes proposiciones son verdaderas (V) o falsas (F). Argumente su respuesta en cada caso.}
        \begin{enumerate}
            \item \textbf{F} La menor cantidad de nodos necesarios para construir un árbol AVL de altura 6 es 34. \\
                La menor cantidad de nodos necesarios para construir un AVL de altura h se puede expresar como:\\
                \[N(x) = N(h-1) + N(h-2) + 1, N(0) = 1, N(1) = 2\]
                \[N(6) = N(5) + N(4) + 1\]
                \[N(6) = 20 + 12 + 1\]
                \[N(6) = 33\]
            \item \textbf{V} La mayor altura de un árbol AVL con 10 nodos es 3.\\
                Utilizando la misma fórmula del inciso anterior podemos ver que la mínima cantidad de nodos necesaria para
                obtener un AVL de altura 3 es 7, pero para obtener un AVL de altura 4 se necesitan 12 nodos, luego, con 10 
                nodos la mayor altura que puede tener un AVL es 3.
                \[N(3) = 7\]
                \[N(4) = 12\]
            \item \textbf{F} El recorrido entreorden en un árbol AVL es $\theta(n)$.\\
                El Recorrido del AVL, como árbol binario, es $O(n)$.
            
            \item \textbf{V} En todo árbol AVL la diferencia entre la longitud del camino más largo desde la raı́z hasta una hoja
            y la longitud del camino más corto desde la raı́z hasta una hoja es siempre menor o igual a 2\\
                Como En un AVL, la diferencia de alturas de sus sub-árboles derecho e izquierdo es a lo más 1, luego la diferencia
                entre el camino más largo y el más corto va a ser, a lo sumo 1, y $1 \leq 2$
        \section{ Diseñe un algoritmo que dado un árbol binario de búsqueda T determine si es un árbol AVL. La complejidad
        temporal del algoritmo debe ser O(n), donde n es la cantidad de nodos en T }
        
        \end{enumerate}
\end{document}
