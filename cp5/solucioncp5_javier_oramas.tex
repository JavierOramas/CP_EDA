\documentclass[10pt,a4paper]{article}
\usepackage[spanish]{babel}
\usepackage{listings}
\usepackage{color}
\usepackage[utf8]{inputenc}

\definecolor{dkgreen}{rgb}{0,0.6,0}
\definecolor{gray}{rgb}{0.5,0.5,0.5}
\definecolor{mauve}{rgb}{0.58,0,0.82}
\definecolor{pink}{rgb}{1.0,0.0,0.5}

\lstset{
  frame=none,
  language=Python,
  aboveskip=3mm,
  belowskip=3mm,
  showstringspaces=false,
  columns=flexible,
  basicstyle={\small\ttfamily},
  numbers=none,
  numberstyle=\tiny\color{pink},
  keywordstyle=\color{blue},
  commentstyle=\color{gray},
  stringstyle=\color{green},
  breaklines=true,
  breakatwhitespace=true,
  tabsize=3
}
\author{Javier A. Oramas López}
\title{CP-5 AVL}

\begin{document}
    \maketitle

    \section{ Determine si las siguientes proposiciones son verdaderas (V) o falsas (F). Argumente su respuesta en cada caso.}
        \begin{enumerate}
            \item \textbf{F} La menor cantidad de nodos necesarios para construir un árbol AVL de altura 6 es 34. \\
                La menor cantidad de nodos necesarios para construir un AVL de altura h se puede expresar como:\\
                \[N(x) = N(h-1) + N(h-2) + 1, N(0) = 1, N(1) = 2\]
                \[N(6) = N(5) + N(4) + 1\]
                \[N(6) = 20 + 12 + 1\]
                \[N(6) = 33\]
            \item \textbf{V} La mayor altura de un árbol AVL con 10 nodos es 3.\\
                Utilizando la misma fórmula del inciso anterior podemos ver que la mínima cantidad de nodos necesaria para
                obtener un AVL de altura 3 es 7, pero para obtener un AVL de altura 4 se necesitan 12 nodos, luego, con 10 
                nodos la mayor altura que puede tener un AVL es 3.
                \[N(3) = 7\]
                \[N(4) = 12\]
            \item \textbf{F} El recorrido entreorden en un árbol AVL es $\theta(n)$.\\
                El Recorrido del AVL, como árbol binario, es $O(n)$.
            
            \item \textbf{V} En todo árbol AVL la diferencia entre la longitud del camino más largo desde la raı́z hasta una hoja
            y la longitud del camino más corto desde la raı́z hasta una hoja es siempre menor o igual a 2\\
                Como En un AVL, la diferencia de alturas de sus sub-árboles derecho e izquierdo es a lo más 1, luego la diferencia
                entre el camino más largo y el más corto va a ser, a lo sumo 1, y $1 \leq 2$
            \end{enumerate}
        \section{ Diseñe un algoritmo que dado un árbol binario de búsqueda T determine si es un árbol AVL. La complejidad
        temporal del algoritmo debe ser O(n), donde n es la cantidad de nodos en T }
        Partiendo de que si un árbol es AVL, cada sub-árbol va a ser AVL también, definimos un algoritmo recursivo
        que va a verificar que cada uno de los subárboles es AVL, de esta manera, solo entra a cada nodo una vez,
        por tanto, la complejidad del algoritmo será O(n)
        \begin{lstlisting}
            def EsAVL(Nodo nodo, int h):

                hIzq = 0                            
                esAVLIzq, hIzq = EsAVL(nodo.izq, 0) 
                if(not esAVLIzq):                   
                    return False, 0                     
                
                hDer = 0                            
                esAVLDer, hDer = EsAVL(nodo.der, 0) 
                if(not esAVLDer):                   
                    sreturn False, 0

                if(abs(hDer-hIzq) > 1):             
                    return False, 0

                h = max(hIzq, hDer)+1               

                return True, h                      
        \end{lstlisting}
        \section{Dada una lista ordenada L de n elementos diseñe un algoritmo que construya un árbol AVL que posea todos
        los elementos de L. La complejidad temporal del algoritmo debe ser O(n).
        }
        La Idea de este algoritmo es recursivamente ir construyendo el árbol, tomando el elemento en el centro de la lista
        como valor del nodo y luego picando en dos la lista construir los sub-árboles, de este modo solo se recorre cada 
        nodo/elemento de la lista una sola vez
        \begin{lstlisting}
            def CrearAVL(L):
                nodo = Node()
                CrearAVLRec(nodo, L)

            def CrearAVLRec( nodo, L):
                l = len(L)
                if (l > 2):
                    nodo.val = L[l/2]
                    nodo.izq = Node()
                    nodo.der = Node()
                    CrearAVL(nodo.Izq, L[:l/2-1])
                    CrearAVL(nodo.Der, L[:l/2+1])
                
                if (l == 2):
                    nodo.val = L[l/2]
                    nodo.izq = Node()
                    CrearAVL(nodo.Izq, L[:l/2-1])
                
                if (l == 1):
                    nodo.val = L[l/2]

        \end{lstlisting}
    \section{Implemente los métodos Insert(T, key) y Erase(T, key) para árboles AVL. La complejidad temporal de
    cada método debe ser O(log n), donde n es la cantidad de nodos en T . Muestre cómo actualizar correctamente
    la propiedad size de los nodos involucrados luego de una inserción/eliminación.
    }

    Sea AVL una estructura de datos que representa un árbol AVL
    cada Nodo t de AVL tiene las siguientes propiedades:
    \begin{enumerate}
        \item val
        \item left
        \item right
        \item h
    \end{enumerate}

    \begin{lstlisting}
        def Insert(node, key):
            if not node:
                return Node(key)
            elif key < node.val:
                node.left = Insert(node.left, key)
            else:
                node.right = Insert(node.right, key)
            
            return MakeBalanced(node)

        def Delete(node, key)
            if not node:
                return node
            
            if key < mode.val:
                node.left = Delete(node.left, key)
            
            if key > mode.val:
                node.right = Delete(node.right, key)
            
            else:
                if node.left is None:
                    temp = node.right
                    node = None
                    return temp
                
                elif node.right is None:
                    temp = node.left
                    node = None
                    return temp
            temp = getMinValueNodez(node.right)
            node.val = temp.val
            node.right = Delete(node.right, temp.val)

            if node is None:
                return node
        
            return MakeBalanced(node)

        def MakeBalanced(node):

            node.h = 1 + max(getHeight(node.left), getHeight(node.right))
            
            balance = getBalance(node)

            if balance > 1 and key < node.left.val:
                return RotateRight(node)

            if balance < -1 and key > node.right.val:        
                return RotateLeft(node)
         
            if balance > 1 and key > node.left.val:
                node.left = RotateLeft(node.left)        
                return RotateRight(node)
  
            if balance < -1 and key < node.right.val:
                node.right = RotateRight(node.right)
                return RotateLeft(node)
         
            return node            
        
        def RotateLeft(z):
            y = z.right
            T2 = y.left
     
            y.left = z
            z.right = T2
   
            z.h = 1 + max(getHeight(z.left),getHeight(z.right))
            y.h = 1 + max(getHeight(y.left),getHeight(y.right))

            return y
     
    
        def RotateRigh(z):
            y = z.left
            T3 = y.right
     
            y.right = z
            z.left = T3
     
            z.h = 1 + max(getHeight(z.left), getHeight(z.right))
            y.h = 1 + max(getHeight(y.left),getHeight(y.right))
     
            return y
        
        def getHeight(node):
            if not node:
                return 0
            return node.h

        def getBalance(node):
            if not node:
                return 0
            return gerHeight(node.left) - getHeight(node.right)
        
        def getMinValueNode(self, root): 

            if root is None or root.left is None: 
                return root 

            return self.getMinValueNode(root.left)

    \end{lstlisting}

    \section{ Sea la estructura de datos Set que simula el comportamiento de un conjunto con las funciones:\\
    Find(S, v): Devuelve True si $v \in S$, False en caso contrario.\\
    Insert(S, v): Añade el elemento v a S si $v \in S$, hace nada en caso contrario.\\
    Erase(S, v): Elimina el elemento v de S si$ v \in S$, hace nada en caso contrario.\\
    Rank(S, v): $|\{u | u \in S ∧ u < v\}|$ (Devuelve la cantidad de elementos $u \in S $que sean menores que v).\\
    Select(S, k): Devuelve $u \in S $ con exactamente k elementos menores en S, o null si u no existe.\\
    SumLessThan(S, v): Devuelve la suma de todos los elementos u ∈ S que sean menores que v.\\
    Diseñe los métodos de la estructura de datos Set con complejidad temporal $O(log |S|)$.\\
    }

    Sea S una estructura que contiene un elemento AVL con acceso a todos los métodos implementados en el ejercicio 
    anterior

    \begin{lstlisting}
        struct Set:
            AVL node 

            def Find(S,v):
                if S.node == None:
                    return False
                
                if S.node.val == v:
                    return True
                
                if S.node.val < v:
                    return Find(S.node.right, v)
                
                return Find(S.node.left, v)

            def Insert(S,v):
                if Find(S,v):
                    S.Node = Insert(S.node, v)
            

            def Delete(S,v):
                if Find(S,v):
                    S.Node = Delete(S.node, v)
            
            def Rank(S,v):
                return CountLess(S.node, v)
            
            def CountLess(node, v) 
                if node == None:
                    return 0
                
                if node.val < v:
                    return 1 + CountLess(node.left, v)+ CountLess(node.right, v)

                elif root.val > v:
                    return CountLess(node.left, v)
            
            def Select(S,k):
                getElems(S.node, k)


            def SumLessThan(S,v):
                return SumLess(S.node, v, [])

            def CountLess(node, v):
                if node == None:
                    return 0
                
                if node.val < v:
                    return node.val + CountLess(node.left, v)+ CountLess(node.right, v)

                elif root.val > v:
                    return CountLess(node.left, v)

            def getElems(node, k, list):
                if node != None:
                    list = getElems(node.left, k, list)

                    if len(list) == k
                        return list
                    list.append(node.val)

                    list = getElems(node.right, k, list)

            


    \end{lstlisting}

\end{document}
